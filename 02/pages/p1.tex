\newpage
\section{{\Large Transformaciones Lineales}}
% Def: E.V
\textbf{Teorema 12.} Sea $V$ un espacio vectorial de dimensión $n$
sobre el cuerpo $\mathbb{F}$. Para cada par de bases ordenadas $\mathcal{B}$,
$\mathcal{B}'$ de $V$ y $W$, respectivamente, la función qm asigna a
una transformación lineal $T$ su matriz respecto a $\mathcal{B}$, 
$\mathcal{B}'$ es un isomorismo entre el espacio $L(V, W)$ y el espacio
de todas las matrices $m \times n$ sobre el cuerpo $\mathbb{F}$.

Estamos particularmente interesados en la representación por matrices de
las transformaciones lineales de un espacio en si mismo, es decir, de 
los operadores lineales sobre un espacio $V$. En tal caso es más conveniente
usar la misma base ordenada. esto es, hacer $\mathcal{B}=\mathcal{B}'$, y
se dirá simplemente que la matriz que la representa es la matriz de $T$
respecto a la base ordenada $\mathcal{B}$. Como este concepto será muy 
importante para este estudio. repasaremos su definición. Si $T$ es un 
operador lineal sobre un espacio vectorial $V$ de dimensión finita, y
$\mathcal{B}=\{\alpha_{1}, \dots, \alpha_{n}\}$ es una base ordenada de $V$, la matriz de $T$ respecto a 
$\mathcal{B}$ (la matriz de $T$ en la base ordenada $\mathcal{B}$) es la
matriz $n \times n$. $A$ cuyos elementos $A$ están definidos por las ecuaciones
\begin{align}
    T\alpha_{i}&=\sum_{i=1}^{n} A_{ij}\alpha_{j} &&& j=1,2, \dots n.
\end{align}
Se debe recordar siempre que esta matriz que representa a $T$ depende de
la base ordenada $\mathcal{B}$, y que existe una matriz que representa a $T$
en cada base ordenad para $V$. (En transformaciones de un espacio en
otro la matriz depende de do bases ordenadas, una de $V$ y otra de $W$)
Para no olvidar esta dependencia usará la notación
\begin{align*}
    [T]_\mathcal{B}
\end{align*}
para la matriz del operador lineal $T$ en la base ordenada $\mathcal{B}$.
La manera cómo esta matriz y la base ordenada describen a $T$, es que
para cada $\alpha$ de $V$
\begin{align*}
    [T\alpha]_\mathcal{B}=[T]_\mathcal{B}[\alpha]_\mathcal{B}
\end{align*}
\vfill
{\footnotesize{\textcopyright FISAT | Prepared using \LaTeXe}}
%\end{center}