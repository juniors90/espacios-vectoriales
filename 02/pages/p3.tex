\newpage
con lo que se debe tener
\begin{align}\label{eq:defi_mul}
    C_{ij}=\sum_{k=1}^{m}B_{ik}A_{kj}
\end{align}
Ya se había motivado la definición \eqref{eq:defi_mul} para la
multiplicación de matrices por operaciones sobre las filas de
una matriz. Aquí se ha visto que una motivación muy convincente
para la definición se tiene mediante la composición de
transformaciones lineales.

\textbf{Teorema 13.} Sean $V$, $W$ y $Z$ espacios vectoriales de
dimensión finita sobre el cuerpo $\mathbb{F}$. Sea $T$ una
transformación lineal de V en W y U una transformación lineal
de $W$ en $Z$. Si $\mathcal{B}$, $\mathcal{B}'$ y $\mathcal{B}''$
son las bases ordenadas de los espacios $V$, $W$ y $Z$,
respectivamente, y si $A$ es la matriz de $T$ respecto al par
$\mathcal{B}$, $\mathcal{B}'$ y $B$ es la matriz de $U$ respecto
al par $\mathcal{B}'$, $\mathcal{B}''$, entonces la matriz de
la composición $UT$ respecta al par $\mathcal{B}$, $\mathcal{B}''$,
es la matriz producto $C = BA$.

\textbf{Teorema 14.} Sea $V$ un espacio vectorial de dimensión finita
sobre el cuerpo $\mathbb{F}$. Sean
\begin{align*}
    \mathcal{B}   &= \{\alpha_{1}, \alpha_{2}, \dots , \alpha_{n}\}&&y&
    \mathcal{B}'  &= \{\beta_{1}, \beta_{2}, \dots ,\beta_{m}\}
\end{align*}
son bases ordenadas de $V$. Supóngase que $T$ es un operador lineal
sobre $V$. Si $P=[P_{1}, P_{2},\dots, P_{n}]$es la matriz $n \times n$
de columnas $P_{j}=[\alpha_{j}']_{\mathcal{B}'}$, entonces
$$[T]_{\mathcal{B}'} = P^{-1}[T]_{\mathcal{B}}P$$
otra manera, si $U$ es el operador lineal sobre $V$ definido por
$U\alpha_{j}=\alpha_{j}'$, $j=1,2,\dots, n$ entonces
\begin{align*}
    [T]_{\mathcal{B}'} = [U]_{\mathcal{B}}^{-1} [T]_{\mathcal{B}}[U]_{\mathcal{B}}
\end{align*}
\textbf{Definición.} Sean $A$ y $B$ dos matrices (cuadradas) $n \times n$
sobre el cuerpo $\mathbb{F}$. Se dice que $B$ es semejante a $A$ sobre
$\mathbb{F}$ si existe una matriz inversible  $n \times n$ $P$ $A$ sobre
$\mathbb{F}$ tal qué $B = P^{-1}AP$.
