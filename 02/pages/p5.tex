\newpage

Sea $\mathcal{B}=\{\alpha_{1}, \alpha_{2}, \dots, \alpha_{n}\}$ una base de $V$.
Conforme al Teorema 1 existe (para cada $i$) un funcional lineal único $f_{i}$
en $V$ tal que
\begin{equation}
    f_{i}(\alpha_{j})=\delta_{ij}
\end{equation}
De esta forma se obtiene de $\mathcal{B}$ un conjtinto de $n$ funcionales
lineales distintos $\{f_{1}, f_{2}, \dots, f_{n}\}$, sobre $V$. Estos
funcionales son también linealmente independientes, pues supóngase que
\begin{equation}
    f=\sum_{i}^{n}c_{i}f_{i}
\end{equation}
Entonces
\begin{align*}
    f_{i}(\alpha_{j})&=\sum_{i=1}^{n}c_{i}f_{i}(\alpha_{j})\\
                     &=\sum_{i=1}^{n}c_{i}\delta_{ij}\\
                     &=c_{j}.
\end{align*}
En particular, si $f$ es el funcional cero $f(\alpha_{i}) = 0$ para cada $j$ y,
por tanto, los escalares $c$, son todos ceros. Entonces los
$\{f_{1}, f_{2}, \dots, f_{n}\}$, son $n$ funcionales linealmente
independientes, y como se sabe que $V^\ast$ tiene dimensión $n$,
deben ser tales que $\mathcal{B}^\ast = \{f_{1}, f_{2}, \dots, f_{n}\}$
es una base de $V^\ast$. Esta base se llama base dual de $\mathcal{B}$.

\textbf{Teorema 15.} Sea $V$ un espacio vectorial de dimensión finita
sobre el cuerpo $\mathbb{F}$ y sea $\mathcal{B}^\ast = \{f_{1}, f_{2}, \dots, f_{n}\}$
una base de $V$. Entonces existe una única base dual
$\mathcal{B}^\ast=\{f_{1}, f_{2}, \dots, f_{n}\}$ de $V^\ast$ tal que
$f_{i}(\alpha_{j})=\delta_{ij}$. Para cada funcional lineal fsobre
$V$ se tiene
\begin{equation}
    f=\sum_{i=1}^nf(\alpha_{i})f_{i}
\end{equation}
y para cada vector $\alpha$ de $V$ se tiene
\begin{equation}
    \alpha=\sum_{i=1}^nf_{i}(\alpha)\alpha_{i}
\end{equation}