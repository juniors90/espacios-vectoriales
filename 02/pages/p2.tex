\newpage
Sean $V$, $W$ y $Z$ espacios vectoriales sobre el cuerpo $\mathbb{F}$ de
dimensiones $n$, $m$ y $p$, respetctivamente.
Sea $T$ una transformación lineal de $V$ en $W$ y $U$ una transformación
lineal de $W$ en $Z$. Supóngase que se tienen las bases ordenadas
\begin{align*}
    \mathcal{B}   &= \{\alpha_{1}, \alpha_{2}, \dots , \alpha_{n}\}\\
    \mathcal{B}'  &= \{\beta_{1}, \beta_{2}, \dots ,\beta_{m}\}\\
    \mathcal{B}'' &= \{\gamma_{1}, \gamma_{2}, \dots ,\gamma_{p}\}
\end{align*}
de los espacios $V$, $W$ y $Z$, respectivamente. Sea $A$ la matriz de $T$
respecto al par $\mathcal{B}$, $\mathcal{B}'$ y sea $B$ la matriz de $U$ 
respecto al par $\mathcal{B}'$, $\mathcal{B}''$. Es fácil ver ahora que la
matriz $C$ de la transformación $UT$ respecto al par $\mathcal{B}$, $\mathcal{B}''$ 
es el producto de $B$ y $A$. En efecto, si $\alpha$ es cualquier vector de $V$
\begin{align*}
    [T\alpha]_{\mathcal{B}'}&=A[\alpha]_{\mathcal{B}}\\
    [U(T\alpha)]_{\mathcal{B}''}&=B[T\alpha]_{\mathcal{B}'}
\end{align*}
Y así
\begin{align*}
    [(UT)(\alpha)]_{\mathcal{B}''}&=BA[T\alpha]_{\mathcal{B}}
\end{align*}
y luego. por la definición y unicidad de la matriz representante, se debe tener
que $C = BA$. Se puede también ver esto haciendo el cálculo
\begin{align*}
    (UT)(\alpha_{i})&=U(T\alpha_{i}) \\
    &=U\left (\sum_{k=1}^{m}A_{kj}\beta_{k}\right ) \\
    &=\sum_{k=1}^{m}A_{kj}U(\beta_{k}) \\
    &=\sum_{k=1}^{m}A_{kj}\sum_{i=1}^{p}B_{ik}\gamma_{i}\\
    &=\sum_{i=1}^{p} \left (\sum_{k=1}^{m}B_{ik}A_{kj}\right )\gamma_{i}
\end{align*}

