\newpage

\section{{\Large Funcionales lineales}}
Si $V$ es un espacio vectorial sobre el cuerpo $\mathbb{F}$, una
transformación lineal $f$ de $V$ en el cuerpo de escalares
$\mathbb{F}$ se llama también un funcional lineal sobre $V$.

Si se comienza desde el principio, esto quiere decir que $f$ es
una función de $V$ en $F$, tal que
$$f(c\alpha + \beta) = cf(\alpha) +f(\beta)$$
para todos los vectores $\alpha$ y $\beta$ de $V$ y todos
los escalares $c$ de $\mathbb{F}$.

\textbf{Ejemplo 19.} Sea $n$ entero positivo y $\mathbb{F}$ un
cuerpo. Si $A$ es una matriz $n \times n$ con elementos
$\mathbb{F}$, la traza de $A$ es el escalar
$$\mathrm{tr} A = A_{11}+A_{11}+\dots+A_{nn}$$
La función traza es un funcional lineal en el espacio de las 
matrices $\mathbb{F}^{n\times n}$, pues

\begin{align*}
    \mathrm{tr} (cA + B) &= \sum_{i=1}^{n}(cA_{ii} + B_{ii})              \\
                &= c\sum_{i=1}^{n} A_{ii} + \sum_{i=1}^{n} B_{ii}\\
                &= c\mathrm{tr}A + \mathrm{tr}B.
\end{align*}

Si $V$ es un espacio vectorial, el conjunto de todos los funcionales lineales
sobre $V$ forman, naturalmente, un espacio vectorial. Es el espacio $L(V, F)$.
Se designa este espacio por $V^\ast$ y se le llama espacio dual del $V$:
$$V^\ast = L(V, F)$$.
Si $V$ es de dimensión finita se puede obtener una descripción
muy explícita del espacio dual $V^\ast$. Por el Teorema 5 sabemos
algo acerca del espacio $V^\ast$
$$dim V^\ast = dim V.$$