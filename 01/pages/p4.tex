\newpage
%\section{{\Large{ICEFOSS-19 FOSS Conference conducted by FFSC}}}
Una función $T$ de $V$ en $W$ se dice inversible si existe una función
$U$ de $W$ en $V$ tal que $UT$ es la función identidad de $V$ y
$TU$ es la función identidad de $W$.
Si $T$ es inversible, la función $U$ es única y se representa por $T^{-1}$.
Más aún, $T$ es inversible si y, solo si

\begin{itemize}
    \item[$(i)$] $T$ es inyectiva, esto es, $T\alpha = T\beta$ implica $\alpha = \beta$,
    \item[$(ii)$] $T$ es sobreyectiva, esto es, la imagen de $T$ es $W$.
\end{itemize}

\textbf{Teorema 7.} Sean $V$ y $W$ dos espacios vectoriales
sobre el cuerpo $\mathbb{F}$ y sea $T$ una transformación lineal
de $V$ en $W$. Si $T$ es inversible, entonces la función recíproca
$T^{-1}$ es una transformación lineal de $W$ en $V$.

Se dice que la transformación lineal $T$ es no singular si
$T\alpha = 0$ implica $\alpha = 0$, es decir, si el espacio
nulo de $T$ es $\{0\}$. Evidentemente, $T$ es inyectiva si y, solo
si, $T$ es no singular. El alcance de esta observación es que
las transformaciones lineales no singulares son las que preservan
la independencia lineal.

\textbf{Teorema 8.} Sean $T$ una transformación lineal de $V$
en $W$. Entonces $T$ es no singular si, y solo si, $T$ aplica
cada subconjunto linealmente independiente de $V$ sobre un
subconjunto linealmente independiente de $W$.

\textbf{Teorema 9.} Sean $V$ y $W$ espacios vectoriales de
dimensión finita sobre el mismo $\mathbb{F}$ tal que
$dim V = dim W$. Si $T$ es una transformación lineal de $V$
en $W$, las siguientes afirmaciones son equivalentes:

\begin{itemize}
    \item[$(i)$] $T$ es inversible.
    \item[$(ii)$] $T$ es no singular.
    \item[$(iii)$] $T$ es sobreyectiva, es decir, la imagen de $T$ es $W$.
\end{itemize}

\textbf{Teorema 9.bis} Sean $V$ y $W$ espacios vectoriales de
dimensión finita sobre el mismo $\mathbb{F}$ tal que
$dim V = dim W$. Si $T$ es una transformación lineal de $V$
en $W$, las siguientes afirmaciones son equivalentes:

\begin{itemize}
    \item[$(i)$] $T$ es inversible.
    \item[$(ii)$] $T$ es no singular.
    \item[$(iii)$] $T$ es sobreyectiva, es decir, la imagen de $T$ es $W$.
    \item[$(iv)$] Si $\{\alpha_{1}, \alpha_{2}, \dots , \alpha_{n}\}$ es
    una base de $V$, entonces $\{T\alpha_{1}, T\alpha_{2}, \dots ,
    T\alpha_{n}\}$ es una base $W$.
\end{itemize}
