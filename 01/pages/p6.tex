\newpage

\begin{align*}
    T\alpha &= T\left (\sum_{j=1}^{n} x_{j}\alpha_{j}\right )\\
            &= \sum_{j=1}^{n} x_{j}(T\alpha_{j})\\
            &= \sum_{j=1}^{n} x_{j} \sum_{i=1}^{m} A_{ij}\beta_{i}\\
            &= \sum_{j=1}^{n} \left ( \sum_{i=1}^{m} A_{ij}x_{j}\right )\beta_{i}
\end{align*}

Si $X$ es la matriz de las coordenadas de $\alpha$ en la
base ordenada ($\mathcal{B}$ entonces el cálculo anterior
muestra que $AX$ es la matriz de las coordenadas del vector $T\alpha$
en la base ordenada $\mathcal{B}'$, ya que el escalar

$$\sum A_{ij}x_{j}$$

es el elemento de la $i$-ésima fila de la matriz columna $AX$.
Obsérvese también que si $A$ es cualquier matriz $m \times n$
sobre el cuerpo $\mathbb{F}$, entonces

\begin{equation}
    T\left (\sum_{j=1}^{n} x_{j}\alpha_{j}\right )
    =
    \sum_{i=1}^{m} \left (\sum_{j=1}^{n} A_{ij}x_{j} \right )\beta_{j}
\end{equation}

define una transformación lineal $T$ de $V$ en $W$, la matriz
de la cual es $A$, respecto a $\mathcal{B}\mathcal{B}'$.

\textbf{Teorema 11.} Sean $V$ un espacio vectorial de dimensión $n$
sobre el cuerpo  $\mathbb{F}$ y $W$ un espacio vectorial de dimensión
$m$ sobre $\mathbb{F}$. Sean $\mathcal{B}$ y $\mathcal{B}'$ una
base ordenada de $V$ y de $W$ respectivamente. Para cada transformación
lineal $T$ de $V$ en $W$, existe una matriz $m \times n$, $A$,
cuyos elementos pertenecen a $\mathbb{F}$, tal que
$$[T\alpha]_{\mathcal{B}'} = A[\alpha]_{\mathcal{B}}$$
mm todo vector $\alpha$ en $V$. Además, $T \to A$ es una correspondencia
biyectiva entre el conjunto de todas las transformaciones lineales
de $V$ en $W$ y el conjunto de todas las matrices $m \times n$ sobre el
cuerpo $\mathbb{F}$.