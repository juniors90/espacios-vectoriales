\newpage

%\begin{center}
%\centering
%\includegraphics[scale=0.4]{ffsc}
%\end{center}
%\begin{center}
%\begin{tips}
%\begin{tabular}{l}
%{\resizebox{5cm}{!}{ONE}}\\ 
%{\resizebox{5cm}{!}{MILLION}}\\ 
%{\resizebox{5cm}{!}{CODES TO}} \\ 
%{\resizebox{5cm}{!}{THE COMMUNITY}}\
%\end{tabular} 
%\end{tips}
%\end{center}
%\begin{center}
\section{{\Large Transformaciones Lineales}}
% Def: E.V
\textbf{Transformación Lineal}. Sean $V$ y $W$. dos espacios vectoriales
sobre el cuerpo $\mathbb{F}$. Una transformación lineal de $V$ en $W$
es una función $T$ de $V$ en W tal que

$$T(c\alpha + \beta) = cT(\alpha) + T(\beta)$$

todos los vectores $\alpha$ y $\beta$ de $V$ y todos 
los escalares $c$ de $\mathbb{F}$.

\textbf{Teorema 1.} Sea $V$ un espacio vectorial de dimensión
finita sobre el cuerpo $\mathbb{F}$, $\{\alpha_{1}, \alpha _{2},
\dots, \alpha_{n}\}$ una base ordenada de $V$. Sean $W$ un espacio
vectorial sobre el mismo cuerpo $\mathbb{F}$ y $\{\beta_{1},
\beta _{2}, \dots , \beta_{n} \}$, vectores cualesquiera de $W$.
Entonces existe una una transformación lineal de $T$ de $V$ en
$W$ tal que
$$T(\alpha_{i}) = \beta_{i}, \forall i = 1, 2, \dots, n$$

% def: Nucleo e Im
\textbf{Definición.} Sean $V$ y $W$ dos espacios vectoriales sobre el
cuerpo $\mathbb{F}$ y sea $T$ una transformación lineal de $V$ en $W$.
El espacio nulo de $T$ es el conjunto de todos los vectores $\alpha$
de $V$ tales que $T(\alpha) = 0$.

Si $V$ es de dimensión finita, el \textit{rango} de $T$ es la
dimensión de la imagen de $T$ y la \textit{nulidad} de $T$ es
la dimensión del espacio nulo de $T$.

\textbf{Teorema 2.} Sean $V$ y $W$ espacios vectoriales sobre el
cuerpo $\mathbb{F}$ y sea $T$ una transformación lineal de $V$
en $W$. Supóngase que $V$ es de dimensión finita. Entonces
$$rango (T) + nulidad (T) = dim V.$$

\textbf{Teorema 2.} Si $A$ es una matriz $m \times n$ de elementos
en el cuerpo $\mathbb{F}$, entonces $$\text{rango de filas} (A)
= \text{rango de columnas} (A).$$

\vfill
\footnotesize{\textcopyright FISAT | Prepared using \LaTeXe}



%\end{center}