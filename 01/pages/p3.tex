\newpage
\section{{\large Álgebra de las Transformaciones Lineales}}
\textbf{Teorema 4.} Sean $V$ y $W$ espacios vectoriales sobre el
cuerpo $\mathbb{F}$. Sean $T$ y $U$ transformaciones lineales de
$V$ en $W$. La función $(T + U)$ definida por
$$(T + U)(\alpha) = T\alpha + U\alpha$$
es una transformación lineal de $V$ en $W$. Si $c$ es cualquier
elemento de $\mathbb{F}$, la función $(cT)$ definida por
$$(cT)(\alpha) = c(T\alpha)$$
es una transformación lineal de $V$ en $W$. El conjunto de
todas las transformaciones lineales de $V$ en $W$, junto con
la adición y la multiplicación escalar aquí definidas,
es un espacio vectorial sobre el cuerpo $\mathbb{F}$.

\textbf{Teorema 5.} Sea $V$ un subespacio vectorial de dimensión
finita $n$ sobre el cuerpo $\mathbb{F}$ y sea $W$ un espacio
vectorial de dimensión finita $m$ sobre $\mathbb{F}$. Entonces
el espacio $L(V, W)$ es de dimensión finita y tiene dimensión $mn$.

\textbf{Teorema 6.} Sean $V$, $W$ y $Z$ espacios vectoriales sobre
el cuerpo $\mathbb{F}$. Sea $T$ una transformación lineal de $V$ en
$W$ y $U$ una transformación lineal de $W$ en $Z$. Entonces la
función compuesta $UT$ definida por $UT(\alpha) = U(T(\alpha))$ es
una transformación lineal de $V$ en $Z$.

\textbf{Definición.}
Si $V$ es un espacio vectorial sobre el cuerpo $\mathbb{F}$,
un \textbf{operador lineal} sobre $V$ es una transformación
lineal de $V$ en $V$.

\textbf{Lema.} Sea $V$ un espacio vectorial sobre el
cuerpo $\mathbb{F}$, sean $U$, $T_{1}$ y $T_{2}$ operadores
lineales sobre $V$, sea $c$ un elemento de $\mathbb{F}$.
\begin{itemize}
    \item[$(i)$] $IU = UI = U$,
    \item[$(ii)$] $U(T_{1} + T_{2}) = UT_{1} + UT_{2}$, $(T_{1} + T_{2})U = T_{1}U + T_{2}U$,
    \item[$(iii)$] $(c(UT_{1}) = (cU)T_{1} = U(cT_{1})$.
\end{itemize}

\vfill
{\footnotesize{ $L(V, W)$ es el espacio de las transformaciones
lineales de $V$ en $W$ sobre el mismo cuerpo  $\mathbb{F}$.}}