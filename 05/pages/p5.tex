\newpage

\textbf{Corolario.} Sea $V$ un espacio vectorial de dimensión
finita sobre el cuerpo $\mathbb{F}$. Si $L$ es un funcional
lineal en el espacio dual $V^{\ast}$ de $V$, entonces existe
un único vector $v\in V$ tal que $$L(v) = f(v)\text{ para todo }f\in V^{\ast}.$$
\textbf{Corolario.} Sea $V$ un espacio vectorial de dimensión
finita sobre el cuerpo $\mathbb{F}$. Toda base de $V^{\ast}$
es dual de alguna base de $V$.

\section{Transpuesta de una transformación lineal}

\textbf{Teorema.} Sean $V$ y $W$ espacios vectoriales
sobre el cuerpo $\mathbb{F}$. Para toda transformación
lineal $T$ de $V$ en $W$, existe una única transformación
lineal  $T^{T}:W^{\ast}\to V^{\ast}$ tal que $$(T^{T}g)(v)
= g(Tv)$$ para todo $g\in W^{\ast}$ y
$v\in V$.

%Sean $V$ y $W$  dos espacios vectoriales
%sobre el cuerpo $\mathbb{F}$ y $T$ una transformación lineal
%de $V$ en $W$ sobre $\mathbb{F}$. Para cada $f \in W^{\ast}$
%tenemos que $T^{T}:W^{\ast}\to V^{\ast}$ es tal que $T^{T}(f)
%= f\circ T \in V^{\ast}$ (por ser la composición de dos
%trasformaciones lineales).

\textbf{Proposicion.} $T^{T}\in \hom_{\mathbb{F}}(W^{\ast}
,V^{\ast})$.

%\textbf{Definición.} Si $V$ es un espacio vectorial sobre el
%cuerpo $\mathbb{F}$ y $S$ es un subconjunto de $V$, el anulador
%de $S$ es el conjunto $S^{\circ}$ de funcionales lineales $f$
%sobre $V$ tales que $f(v) = 0$ para todo $v$ de $S$.

%\textbf{Teorema.} Sea $V$ un espacio vectorial de dimensión
%finita sobre el cuerpo $\mathbb{F}$ y sea $W$ un subespacio
%de $V$. Entonces
%$$\dim W + \dim W^{\circ} = \dim V.$$
%\textbf{Corolario.} Si $W_1$ y $W_2$ son subespacios de un
%espacio vectorial de dimención finita, entonces $W_1 = W_2$
%si, y solo si, $W_1^{\circ} = W_2^{\circ}$.

\textbf{Teorema.} Sean $V$ y $W$  dos espacios vectoriales
sobre el cuerpo $\mathbb{F}$ y $T$ una transformación lineal
de $V$ en $W$. Entonces $\ker(T^{T})=Im(T)^{\circ}$.
Particularmente, si $\dim V, \dim W < \infty$ entonces
\begin{itemize}
    \item[$(a)$] $rango(T)=rango(T^{T})$.
    \item[$(b)$] $Im(T^{T})=(\ker T)^{\circ}$.
\end{itemize}

\textbf{Teorema.} Sean $V$ y $W$  dos espacios vectoriales
sobre el cuerpo $\mathbb{F}$ y $\dim(V) = m, \dim(W) = n$,
con $m,n < \infty$ tales que $\mathcal{B}_1=\{v_1,v_2,\dots,v_m\}$ y
$\mathcal{B}_2=\{w_1,w_2,\dots,w_n\}$ sean las bases de $V$ y
$W$, respectivamente. Además $\mathcal{B}_1^{\ast}=\{f_1,f_2,
\dots,f_m\}$ y $\mathcal{B}_2^{\ast}=\{g_1,g_2,\dots,g_n\}$
son las bases duales de $\mathcal{B}_1$ y $\mathcal{B}_2$,
respectivamente.  
\begin{align*}
    \mathcal{B},\mathcal{C}&\to [T]_{\mathcal{B}\mathcal{C}}
    \in \mathbb{F}^{m\times n}\\
    \mathcal{B}^{\ast},\mathcal{C}^{\ast}&\to [T^{T}]_{\mathcal
    {C}^{\ast}\mathcal{B}^{\ast}}\in\mathbb{F}^{n\times m}
\end{align*}
con $T \in \hom_{\mathbb{F}}(V,W)$ entonces
$[T^{T}]_{\mathcal{C}^{\ast}\mathcal{B}^{\ast}}
=[T]_{\mathcal{B}\mathcal{C}}^{T}$

