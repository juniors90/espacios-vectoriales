\newpage
\section{{\Large Operadores Semisimples}}

Sea $\mathbb{F}=\mathbb{R}$ ó $\mathbb{C}$

\textbf{Definición.} Sea $V$ un espacio vectorial de dimensión
finita sobre el cuerpo $\mathbb{F}$ y sea $T$ un operador
lineal sobre $V$. Se dice que $T$ es \textbf{semisimple} si
todo subespacio $T$-invariante tiene un subespacio
complementario $T$-invariante.

\textbf{Lema.} Sea $T$ un operador lineal sobre el espacio
vectorial de dimensión finita $V$ y sea $V = W_1\oplus W_2 \oplus
\dots \oplus W_k$ la descomposición prima de $T$; es decir,
si $m_T$ es el polinomio minimal de $T$ y $m_T = p_1^{r_1}p_2^{r_2}
\dots p_k^{r_k}$ es la factorización prima de $m_T$,
entonces es $W_j$ el espacio nulo de $p_j(T)^{r_j}$. Sea $W$ el
subespacio de $V$ que es invariante por $T$. Entonces
$$W=(W\cap W_1)\oplus (W\cap W_2)\oplus \dots (W \cap W_k) $$

\textbf{Lema.} Sea $T$ un operador lineal sobre el espacio
vectorial $V$ y sea $m_T$ el polinomio minimal de $T$. Si $m_T$
es irreducìble sobre el cuerpo escalar $\mathbb{F}$ entonces
$T$ es semisimple.

\textbf{Teorema} Sea $T$ un operador lineal sobre un espacio
rectorial de dimensión finita $V$. Una condición necesaria y
suficiente para que $T$ sea semisimple es que el polinomio
minimal $m_T$ de $V$ y sea de la forma $p = p_1 p_2 \cdots
p_k$, donde $p_1, p_2, \dots, p_k$ son polinomios irreducibles
distintos sobre el cuerpo escalar $\mathbb{F}$.

\textbf{Corolario.} Si $T$ es un operador lineal sobre un
espacio vectorial de dimensión finita sobre un cuerpo
algebraicamente cerrado, $T$ es semisimple si, y solo si,
$T$ es diagonalizable.

\section{Espacio Dual}

Si $V$ es un espacio vectorial, el conjunto de todos los
funcionales lineales sobre $V$ forman, naturalmente, un
espacio vectorial. Es el espacio $L(V, F)$.

Se designa este espacio por $V^{\ast}$ y se le llama
\textbf{espacio dual} del $V$: $$V^{\ast} = L(V, F)$$.