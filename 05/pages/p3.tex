\newpage

Si $V$ es de dimensión finita se puede obtener una descripción
muy explícita del espacio dual $V^{\ast}$.

Por el Teorema 5 sabemos algo acerca del espacio $V^{\ast}$:
$$\dim V^{\ast} = \dim V.$$
Sea $\mathcal{B} = \{v_1,v_2,\dots, v_n\}$ una base de $V$.
Conforme al Teorema 1 existe (para cada $i$) un funcional
lineal único $f_i$ en $V$ tal que $$f_i(v_j)=\delta_{ij}.$$
De esta forma se obtiene de $\mathcal{B}$ un conjunto de $n$
funcionales lineales distintos $f_1,\dots , f_k$, sobre $V$.
Estos funcionales son también linealmente independientes,
pues supóngase que
\begin{align*}
    f_i = \displaystyle{\sum_{i=1}^{n}}c_if_i
\end{align*}
Entonces
\begin{align*}
    f(v_j) &= \displaystyle{\sum_{i=1}^{n}}c_if_i(v_j)\\
           &= \displaystyle{\sum_{i=1}^{n}}c_i\delta_{ij}\\
           &= c_j.
\end{align*}
En particular, si $f$ es el funcional cero. $f_i(v_j) = 0$ para
cada $j$ y, por tanto, los escalares $c$, son todos ceros.
Entonces los $f_1,\dots, f_k$, son $n$ funcionales linealmente
independientes, y como se sabe que $V^{\ast}$ tiene dimensión $n$,
deben ser tales que $\mathcal{B}^{\ast} = \{f_1,\dots, f_k\}$
es una base de $V^{\ast}$.
Esta base se llama \textbf{base dual} de $\mathcal{B}$.

\textbf{Teorema.} Sea $V$ un espacio vectorial de dimensión
finita sobre el cuerpo $\mathbb{F}$ y sea $\mathcal{B} =
\{v_1,v_2,\dots,v_n\}$ una base de $V$. Entonces existe una
única base dual $\mathcal{B}^{\ast} = \{f_1,f_2,\dots,f_n\}$
de $V^{\ast}$ tal que $f_i(v_j) = \delta_{ij}$. Para cada
funcional lineal $f$ sobre $V$ se tiene
\begin{align*}
    f(v) &= \displaystyle{\sum_{i=1}^{n}}f_i(v_i)f_i
\end{align*}
y para cada vector $v$ de $V$ se tiene
\begin{align*}
    v &= \displaystyle{\sum_{i=1}^{n}}f_i(v)v_i
\end{align*}