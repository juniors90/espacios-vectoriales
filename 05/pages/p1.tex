\newpage

Si queremos dar una forma canónica debemos pensar en cómo
describir el $N$. Por corolario anterior, si $N$ es nilpotente,
existe $r>0$ tal que $N^{r}=0$ entonces $m_N|x^{r}$. Con lo cual
$N$ es nilpotente si y solo sí
$m_N=x^{s}$ para algún $s\in \mathbb{N}, s\leq n$ si y solo sí
$p_N=x^{n}$ para algún $n=\dim(V)$.

Por \textbf{teorema de descomposición cíclica}, tenemos que si
$$
p_r=x^{s_r},
p_{r-1}=x^{s_{r-1}},
\dots
p_2=x^{s_2},
p_1=x^{s_1}=m_N
$$
con $s_r \leq s_{r_1} \leq \dots \leq s_2 \leq s_1$ tales que
$s_r + s_{r_1} + \dots + s_2 + s_1 =\dim(V)=n$. Existe una base
$\mathcal{B}$ tal que
\begin{equation*}[N]=\left(
	\begin{smallmatrix}
		0      & 0      & \cdots & 0      & 0      &        &          &        &        &        &        &\\
		1      & 0      & \cdots & 0      & 0      &        &          &        &        &        &        &\\
		0      & 1      & \cdots & 0      & 0      &        &          &        &        &        &        &\\
		\vdots & \vdots & \ddots & \vdots & \vdots &        &          &        &        &        &        &\\
		0      & 0      & \cdots & 1      & 0      &        &          &        &        &        &        &\\
		&        &        &        &        & 0      & 0      & 0        & \cdots & 0      & 0      &\\
		&        &        &        &        & 1      & 0      & 0        & \cdots & 0      & 0      &\\
		&        &        &        &        & 0      & 1      & 0        & \cdots & 0      & 0      &\\
		&        &        &        &        & 0      & 0      & 1        & \cdots & 0      & 0      &\\
		&        &        &        &        & \vdots & \vdots &          & \ddots & \vdots & \vdots &\\
		&        &        &        &        & 0      & 0      &          & \cdots & 1      & 0      &\\
		&        &        &        &        &        &        &          &        &        &        &\ddots 
	\end{smallmatrix}\right)
\end{equation*}

Recíprocamente, las matrices de este tipo dan operadores nilpotentes.

\textbf{Observación.} En general, a menos de equivalencia,
hay tantas matrices de tamaño $n$ como lo que se llama particiones
de $n$ formas de cubrir a $n$ como suma de números naturales.


\textbf{Definición.} Una matriz bloque elemental de Jordan
con valor propio $c$ (asociada a un escalar ``$c$'') es la
matriz $B \in \mathbb{F}^{n\times n}$ forma:
\begin{equation}B=\left(
	\begin{matrix}
		c      & 0      & \cdots & 0      \\
		0      & c      & \cdots & 0      \\
		\vdots & \vdots & \ddots & \vdots \\
		0      & 0      & \cdots & c      
	\end{matrix}\right)
\end{equation} 

\textbf{Teorema.} Sea $V$ un espacio vectorial de dimensión finita
y sea $T:V \to V$ tal que $m_T$ es producto de factores lineales
entonces existe $\mathcal{B}$ una base de $V$ tal que $[T]_{\mathcal{B}}$
es un bloque de Jordan. La descomposición es única con respecto
al tamaño de cada bloque.