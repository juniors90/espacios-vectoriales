\newpage

\textbf{Definición.} Si $V$ es un espacio vectorial sobre el
cuerpo $\mathbb{F}$ y $S\subseteq V$, el anulador
de $S$ es el conjunto $S^{\circ}=\{f:V\to\mathbb{F}: f(v) = 0
\text{ para todo }v\in S\}$.

\textbf{Teorema.} Sea $V$ un espacio vectorial de dimensión
finita sobre el cuerpo $\mathbb{F}$ y sea $W$ un subespacio
de $V$. Entonces $$\dim W + \dim W^{\circ} = \dim V.$$
\textbf{Corolario.} Si $W_1$ y $W_2$ son subespacios de un
espacio vectorial de dimención finita, entonces $W_1 = W_2$
si, y solo si, $W_1^{\circ} = W_2^{\circ}$.

\section{Doble Dual}

Una pregunta es, ¿si toda base
de $V^{\ast}$ es la dual de alguna base de $V$?.
Una posibilidad de contestar esta pregunta es considerar
$V^{\ast\ast}$, espacio dual de $V^{\ast}$.

Si $v\in V$, entonces $v$ induce un funcional lineal $L_{v}$,
sobre $V^{\ast}$, definido por
$$L_{v}=f(v)$$
El hecho de que $L_{v}$, sea lineal no es más que una
reformulación de la definición de las operaciones lineales
en $V^{\ast}$:
\begin{align*}
    L_v(cf + g) &= (cf + g)(v)\\
                &= (cf)(v) + g(v)\\
                &= cf(v) + g(v)\\
                &= cL_v(f) + L_v(g)
\end{align*}
Si $V$ es de dimensión finita y $v \not = 0$, entonces
$L_v \not = 0$. Es decir, existe un funcional
lineal $f$ tal que $f(v) \not = 0$.

La demostración es muy simple: Elíjase una base ordenada
$\mathcal{B} = \{v_1, v_2, \dots, v_n\}$ de $V$ tal que
$v_1 = v $, y sea $f$ el funcional lineal que asigna a cada
vector en $V$ su primera coordenada en la base ordenada 
$\mathcal{B}$.

\textbf{Teorema.} Sea $V$ un espacio vectorial de dimensión
finita sobre el cuerpo $\mathbb{F}$. Para cada vector $v\in V$
se define
\begin{align*}
    L_v(f) &= f(v)\text{ con $f\in V^{\ast}$}.
\end{align*}
Entonces $L:V\to V^{\ast\ast}$ tal que $L(v)=L_v$
es un isomorfismo.
