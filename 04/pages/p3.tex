\newpage

\section{Forma de Jordan}

Supongamos que $T$ es un operador lineal sobre $V$ y
que el polinomio característico de $T$ se puede factorizar
sobre $\mathbb{F}$ como sigue $p_T=(x-c_1)^{r_1}\dots (x-c_k)^{r_k}$,
con $r_i>0$, $c_i\not = c_j$. Entonces el polinomio minimal
de $T$ será $m_T=(x-c_1)^{a_1}\dots (x-c_k)^{a_k}$
con $0<a_i\leq r_k$

Si $V_i$ es el espacio nulo de $(T-c_iI)^{a_i}$, entonces
el teorema de descomposición prima dice que
$$V=V_1\oplus V_2 \oplus \dots \oplus V_k$$
y que el operador $T$, inducido en $V_i$ por $T$ tiene
polinomio minimal $(x - c_i)^{a_i}$.

Esto nos da el puntapié para hallar la descomposición de $T$
con $T=D+N$.

\textbf{Observación:} La desomposición de $D$ es
$$D = c_1E_1 + c_2E_2 + \dots + c_kE_k$$

Si queremos dar una forma canónica debemos pensar en cómo
describir el $N$. Por corolario anterior, si $N$ es nilpotente,
existe $r>0$ tal que $N^{r}=0$ entonces $m_N|x^{r}$. Con lo cual
$N$ es nilpotente si y solo sí
$m_N=x^{s}$ para algún $s\in \mathbb{N}, s\leq n$ si y solo sí
$p_N=x^{n}$ para algún $n=\dim(V)$.

Por \textbf{teorema de descomposición cíclica}, tenemos que si
$$
p_r=x^{s_r},
p_{r-1}=x^{s_{r-1}},
\dots
p_2=x^{s_2},
p_1=x^{s_1}=m_N
$$
con $s_r \leq s_{r_1} \leq \dots \leq s_2 \leq s_1$ tales que
$s_r + s_{r_1} + \dots + s_2 + s_1 =\dim(V)=n$. Existe una base
$\mathcal{B}$ tal que
\begin{equation*}[N]=\left(
\begin{smallmatrix}
0      & 0      & \cdots & 0      & 0      &        &          &        &        &        &        &\\
1      & 0      & \cdots & 0      & 0      &        &          &        &        &        &        &\\
0      & 1      & \cdots & 0      & 0      &        &          &        &        &        &        &\\
\vdots & \vdots & \ddots & \vdots & \vdots &        &          &        &        &        &        &\\
0      & 0      & \cdots & 1      & 0      &        &          &        &        &        &        &\\
       &        &        &        &        & 0      & 0      & 0        & \cdots & 0      & 0      &\\
       &        &        &        &        & 1      & 0      & 0        & \cdots & 0      & 0      &\\
       &        &        &        &        & 0      & 1      & 0        & \cdots & 0      & 0      &\\
       &        &        &        &        & 0      & 0      & 1        & \cdots & 0      & 0      &\\
       &        &        &        &        & \vdots & \vdots &          & \ddots & \vdots & \vdots &\\
       &        &        &        &        & 0      & 0      &          & \cdots & 1      & 0      &\\
       &        &        &        &        &        &        &          &        &        &        &\ddots 
    \end{smallmatrix}\right)
\end{equation*}

