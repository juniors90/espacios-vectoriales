\newpage

\textbf{Teorema.} (Teorema de descomposición cíclica).
Sea $T$ un operador lineal lineal sobre un espacio vectorial
de dimensión finita y sea $W_0$ un subespacio propio
$T$-admisible de $V$. Existen los vectores no nulos
$v_1, v_2, \dots v_r\in V$ con $T$-anuladores
$p_i=m_{v_i, T}$, $i=1,2,\dots,r$ tales que
\begin{itemize}
    \item[$i)$] $V=W_0\oplus Z(v_1, T)\oplus \dots \oplus Z(v_k, T)$.
    \item[$ii)$] $p_k|p_{k-1}, k=2,\dots,r$.
\end{itemize}

\textbf{Corolario.} Si $T$ es un operador lineal sobre
un espacio vectorial de dimensión finita entonces todo
subespacio $T$-admisible tiene un subespacio complementario
que es también invariante por $T$.

\textbf{Corolario.}
Sea $T$ un operador lineal sobre un espacio vectoría $V$ de dimensión
finita
\begin{itemize}
    \item[$i)$] Existe un vector $v\in V$ tal que el $T$-anulador
    de $v$ es el polinomio minimal de $T$.
    \item[$ii)$] $T$ tiene un vector cíclico si, y solo si, los
    polinomios característico y minimal de $T$ son idénticos.
\end{itemize}

\textbf{Teorema.} (Teorema de Cayley-Hamilton generalizado). Sea $T$
un operador lineal sobre un espacio vectorial $V$ de dimensión finita.
Sean $m_T$ y $p_T$ los polinomios minimal y característico de $T$,
respectivamente.
\begin{itemize}
    \item[$i)$]  $m_T$ divide a $p_T$.
    \item[$ii)$]  $m_T$ y $p_T$ tienen los mismos factores primos,
    salvo multiplicidades.
    \item[$iii)$] Si $$m_T=p_1^{r_1}p_2^{r_2}\dots p_k^{r_k}$$
    es la factorización prima de $p_T$, entonces
    $$p_T=p_1^{d_1}p_2^{d_2}\dots p_k^{d_k}$$
    donde $d_i$ es la nulidad de $p_i(T)^{r_i}$ dividida por el grado de $p_i$.
\end{itemize}

\textbf{Corolario.} Si $T$ es un operador lineal nilpotente
sobre un espacio vectorial de dimensión $n$, entonces el
polinomio característico de $T$ es $x^{n}$.

\textbf{Teorema.} Sea $\mathbb{F}$ un cuerpo y sea $B\in
\mathbb{F}^{n\times n}$. Entonces $B$ es semejante sobre
$\mathbb{F}$ a una, y solo a una, matriz que está en forma
racional.