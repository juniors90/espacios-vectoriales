\newpage



\textbf{Definición.} Sea $A \in \mathbb{F}^{n\times n}$. $A^{\ast}
=\overline{[A]^{t}}$ es la traspuesta conjugada de $A$.

\textbf{Teorema.} Sea $V$ un espacio producto interno de dimensión
finita. Si $T$ y $U$ son operadores lineales sobre $V$ y $c$ es un
escalar
\begin{itemize}
    \item[$i)$] $(T + U)^{\ast}= T^{\ast} + U^{\ast}$.
    \item[$ii)$] $(cT)^{\ast} = \overline{c}T^{\ast}$.
    \item[$iii)$] $(TU)^{\ast} = U^{\ast}T^{\ast}$.
    \item[$iv)$] $(T^{\ast})^{\ast} = T$.
\end{itemize}

\section{Operadores Unitarios}

\textbf{Definición.} Sean $V$ y $W$ espacios producto interno sobre
el mismo cuerpo y sea $T$ una transformación lineal de $V$ en $W$.
Se dice que $T$ preserva productos internos si $\langle T(v)|T(w)
\rangle = (v|w)$ para todo $v$, $w$ de V. Un isomorfismo de $V$
sobre $W$ es un isomorfismo $T$ de espacio vectorial de $V$ sobre
$W$ que también preserva productos internos.

Si $T$ preserva productos internos, entonces $||T(v)||=||v||$ y así $T$
es necesariamente no singular.
Así que un isomorfismo de $V$ sobre $W$ puede ser definido
también como una transformación lineal de $V$ sobre $W$ que preserva
productos internos. Si Tes un isomorfismo de $V$ sobre $W$, entonces
$T^{-1}$ es un isomorfismo de $W$ sobre $V$; luego, cuando tal $T$
existe, se dirá simplemente que $V$ y $W$ son isomorfos.
Naturalmente, el isomorfo de espacio producto interno es una relación
de equivalencia.

\textbf{Teorema.} Sean $V$ y $W$ espacios producto interno de
dimensión finita sobre el mismo cuerpo y que tienen la misma
dimensión. Si $T$ es una transformación lineal de $V$ en $W$,
las siguientes afirmaciones son equivalentes.
\begin{itemize}
	\item[$(i)$] $T$ preserva los productos internos.
	\item[$(ii)$] $T$ es un isomorfismo (en un espacio
	producto interno).
	\item[$(iii)$] $T$ aplica toda base ortonormal de $V$
	sobre una base ortonormal de $W$.
	\item[$(iv)$] $T$ aplica cierta base ortonormal de $V$
	sobre una base ortonormal de $W$.
\end{itemize}