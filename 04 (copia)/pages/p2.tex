\newpage


\textbf{Definición.} Sean $V$ un Espacio Vectorial y $u, v \in V$.

\begin{itemize}
    \item[$i)$] $u$ y $v$ son \textbf{ortogonales} si $\langle u|v \rangle = 0$
    \item[$ii)$] $S \subseteq V$ subconjunto es un \textbf{conjunto ortogonal} si
                 $\langle u|v \rangle = 0, \forall u, v \in S, u \not = v$.
    \item[$iii)$] $S \subseteq V$ subconjunto es un \textbf{conjunto ortonormal} si 
                 es ortogonal y $||v|| = 1, \forall v \in S$.
    \item[$iv)$] Una base $\mathcal{B}$ es \textbf{ortogonal} (\textbf{ortonormal})
                 si lo es como conjunto.
\end{itemize}

\textbf{Observación.} Sea $S$ un \textbf{conjunto ortogonal} tal que $0 \not\in S$ entonces
$S'=\left\{ \frac{v}{||v||}:v \in S \right\}$ es un \textbf{conjunto ortonormal} 

\textbf{Teorema.} Sea $S \subseteq V$ un \textbf{conjunto ortogonal}
tal que $0 \not\in S$ entonces $S$ es \textbf{linealmente independiente}.

\textbf{Teorema.}

Sean $V$ un espacio vectorial con producto interno $\{v_{1},
v_{2}, \dots, v_{n}\}$ en $V$ \textbf{linealmente independientes}
entonces existen $\{w_{1}, w_{2}, \dots, w_{n}\}$ ortogonales tales que

$\langle v_{1}, v_{2}, \dots, v_{k} \rangle=\langle w_{1}, w_{2}, \dots, w_{k} \rangle,~k=1,2,\dots, n.$

IDEA:

$w_{1}=v_{1}$;

$w_{j}=v_{j}-\displaystyle{\sum_{i=0}^{j-1}}\frac{\langle v_{j}|w_{i} \rangle}{||w_{i}||^2}$, con $j>1$.

\textbf{Corolario.}

Todo espacio vectorial, de dimensión finita, con producto interno tiene bases
ortogonales.

\textbf{Observación.} Sean $V$ un espacios vectorial sobre el 
cuerpo $\mathbb{F}$ y $\mathcal{B}=\{v_{1},\dots,v_{n}\}$ una base de $V$.
$f_{i}:V\to \mathbb{F}$
\begin{align*}
	f_{i}(v) = \frac{\langle v_{j}|w_{i} \rangle}{||w_{i}||^2}
	\Rightarrow f_{i}=\delta_{ij}
\end{align*}
para todo $i, j = 1, 2, \dots, n$.

$\mathcal{B}^{\ast}=\{f_{1}, f_{2}, \dots, f_{n}\}$ es la base dual de $\mathcal{B}$
entonces 
\begin{align*}
	v=\displaystyle{\sum_{i=0}^{n}} \frac{\langle v|v_{i} \rangle}{||v_{i}||^2}v_{i}.
\end{align*}
