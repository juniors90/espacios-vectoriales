\newpage
\textbf{Teorema 21.} Sean $V$ y $W$ espacios vectoriales sobre el 
cuerpo $\mathbb{F}$. Para toda transformación lineal $T$ de $V$
en $W$, existe una única transformación lineal $T^{t}$ de $W^{\ast}$
en $V^{\ast}$ tal que
\begin{align*}
T^{t}g(\alpha) = g(T\alpha)
\end{align*}
para todo $g$ de $W^{\ast}$ y todo $\alpha$ de $V$.

A $T^{t}$ se la llama \textbf{transpuesta} de $T$. Esta transformación
$T^{t}$ también se llama a menudo adjunta de $T$, pero no usaremos
esta terminología.

\textbf{Teorema 22.} Sean $V$ y $W$ espacios vectoriales sobre
el cuerpo $\mathbb{F}$ y sea $T$ una transformación lineal de
$V$ en $W$. El espacio nulo de $T^{t}$ es el anulador de la imagen
de $T$. Si $V$ y $W$ son de dimensión finita, entonces
\begin{itemize}
    \item[$(i)$] $\mathtt{rango} (T^{t}) = \mathtt{rango} (T)$
    \item[$(ii)$] la imagen de $T^{t}$ es el anulador del espacio
    nulo de $T$.
\end{itemize}
\textbf{Teorema 23.} Sean $V$ y $W$ espacios vectoriales de dimensión
finita sobre el cuerpo $\mathbb{F}$. Sea $\mathcal{B}$ una base ordenada
de $V$ con base dual $\mathcal{B}^{\ast}$, y sea $\mathcal{B}'$ una base
ordenada de $W$ con base dual $\mathcal{B}'^{\ast}$. Sea $T$ una
transformación lineal de $V$ en $W$.

Sea $A$ la matriz de $T$ respecto a $\mathcal{B}, \mathcal{B}'$  y sea
$B$ la matriz de $T^{t}$ respecto a $\mathcal{B}'^{\ast},\mathcal{B}^{\ast}$
Entonces $B_{ij}=A_{ji}$

\textbf{Definición.} Si $A$ es una matriz $m\times n$ sobre el cuerpo
$\mathbb{F}$, la transpuesta de $A$ es la matriz $n\times m$,
$A^{t}$, definida por $A^{t}_{ij} = A_{ji}$.

El Teorema 23 dice, pues, que si $T$ es una transformación lineal de
$V$ en $W$, cuya matriz con respecto a un par de bases es $A$, entonces
la transformación transpuesta $T^{t}$ está representada, en el par de
bases dual, por la matriz transpuesta $A^{t}$.

\textbf{Teorema 24.} Sea $A$ cualquier matriz $m\times n$ sobre el
cuerpo $\mathbb{F}$. Entonces el rango de filas de $A$ es igual al
rango de columnas de $A$.