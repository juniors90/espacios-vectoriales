\newpage
El hecho de que $L_{\alpha}$, sea lineal no es más que una reformulación
de la definición de las operaciones lineales en $V^{\ast}$:
\begin{align*}
    L_{\alpha}(cf + g) &= (cf + g)(\alpha)\\
    &= (cf)(\alpha) + g(\alpha)\\
    &= cf(\alpha) + g(\alpha)\\
    &= cL_{\alpha}(f) + L_{\alpha}(g).
\end{align*}

Si $V$ es de dimensión finita y $\alpha not = 0$, entonces $L_{\alpha}$,
en otras palabras, existe un funcional lineal $f$ tal que $f(\alpha)$.
La demostración es muy simple y fue dada en la Sección 3.5:
Elíjase una base ordenada $\mathcal{B}=\{\alpha_{1}, \alpha_{2},
\dots, \alpha_{n}\}$ de $V$ tal que $\alpha_{1}=\alpha$,
y sea $f$ el funcional lineal que asigna a cada vector en $V$
su primera coordenada en la base ordenada $\mathcal{B}$.

\textbf{Teorema 17.} Sea $V$ un espacio vectorial de dimensión
finita sobre el cuerpo $\mathbb{F}$. Para cada vector $\alpha$ de $V$ se define
\begin{align*}
    L_{\alpha}(f)=f(\alpha),  f \text{ en }V^{\ast}.
\end{align*}
La aplicación $\alpha \to L_{\alpha}$ es entonces un isomorfismo de $V$
sobre $V^{\ast\ast}$.

\textbf{Corolario.} Sea $V$ un espacio vectorial de dimensión finita sobre
el cuerpo $\mathbb{F}$. Si $L$ es un funcional lineal en el espacio dual
$V^{\ast}$ de $V$, entonces existe un único rector $\alpha$ de $V$ tal que
$$L(f) = f(\alpha)$$
para  todo $f$ de $V^{\ast}$.

\textbf{Corolario.} Sea $V$ un espacio vectorial de dimensión
finita sobre el cuerpo $\mathbb{F}$. Toda base de $V^{\ast}$
es dual de alguna base de $V$.

\textbf{Teorema 18.} Si $S$ es cualquier subconjunto de un espacio
vectorial de dimensión finita $V$, entonces $(S^{\circ})^{\circ}$
es el subespacio generado por $S$.

\textbf{Definición}. Si $V$ es un espacio vectorial, un hiperespacio
en $V$ es un subespacio propio maximal de $V$.
