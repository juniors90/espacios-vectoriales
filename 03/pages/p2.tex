\newpage
\textbf{Teorema 19.} Si $f$ es un funcional lineal $\alpha$ nulo
sobre el espacio vectorial $V$, entonces el espacio nulo de $f$
es un hiperespacio en $V$. Recíprocamente, todo hiperespacio de $V$
es el espacio nulo de un (no único) funcional lineal no nulo
sobre $V$.

\textbf{Lema.} Si $f$ y $g$ son funcionales lineales en el
espacio vectorial $V$, entonces $g$ es un múltiplo escalar
de $f$ si y solo si, el espacio nulo de $g$ contiene al espacio
nulo de $f$, esto es, si y solo si, $f(\alpha) = 0$ implica
que $g(\alpha) = 0$.

\textbf{Teorema 20.} Sean $g, f_{1},\dots, f_{n}$ funcionales lineales
sobre un espacio vectoriales $V$ con espacios nulos $N, N_{1}, N_{2},\dots, N_{r}$
respectivamente. Entonces $g$ es  una combinación lineal de los
$f_{1},\dots, f_{n}$ si y solo si $N$ contiene la intersección
$N_{1}\cap N_{1}\cap\dots\cap N_{r}$

\section{{\Large Transpuesta de una transformación lineal}}

Supóngase que se tienen dos espacios vectoriales $V$ y $W$ sobre
el cuerpo $\mathbb{F}$ y una transformación lineal $T$ de $V$ en $W$.
Entonces $T$ induce una transformación lineal de $W^{\ast}$ en
$V^{\ast}$, como sigue. Supóngase que $g$ es un funcional lineal
en $W$ y sea
\begin{equation}\label{eq:fun_lin}
    f(\alpha)=g(T\alpha)
\end{equation}
para cada $\alpha$ en $V$. Entonces \eqref{eq:fun_lin} define una
función $f$ de $V$ en $\mathbb{F}$, que es la composición de $T$,
función de $V$ en $W$, con $g$, función de $W$ en $\mathbb{F}$.
Como ambas, $T$ y $g$, son lineales, el Teorema 6 dice que $f$
es también lineal; vale decir, $f$ es una función lineal en $V$.

Así $T$ suministra una correspondencia $T^{t}$ que asocia a
cada funcional lineal $g$ sobre $W$ un funcional lineal
$f = T^{t}g$ sobre $V$, definido por \eqref{eq:fun_lin}.
Obsérvese también que $T^{t}$ es igualmente una transformación
lineal de $W^{\ast}$. En efecto, si $g_{1}$ y $g_{2}$ están en
$W^{\ast}$ y $c$ es un escalar
\begin{align*}\label{eq:fun_lin_tras}
    [T^{t}(cg_{1} + g_{2})](\alpha)&=(cg_{1} + g_{2})(T\alpha)\\
    &=cg_{1}(T\alpha) + g_{2}(T\alpha)\\
    &=c(T^{t}g_{1})(\alpha) + (T^{t}g_{2})(\alpha)
\end{align*}
de modo que $T^{t}(cg_{1} + g_{2}) = cT^{t}g_{1} + T^{t}g_{2}$.